\documentclass[10pt,a4paper]{article}
\usepackage[mathletters]{ucs}
\usepackage[utf8x]{inputenc}
\usepackage{geometry}
\usepackage{amsmath}
\usepackage{amsfonts}
\usepackage{amssymb}
\usepackage{verbatim}
\usepackage{proof}
\usepackage{eurosym}
\usepackage{listings}
\usepackage{tikz}
\usepackage{syntax}
\usetikzlibrary{arrows,shapes,automata}
%\geometry{a4paper,left=20mm,right=20mm, top=2cm, bottom=2cm}

\begin{comment}
nützliche befehle
\stackrel{oben}{unten}
\mathcal \mathfrak \mathbb
\operatorname{operator}
\newcommand{\makroname}{ was das makro ausgibt }
\infer[name]{unten}{oben}
\end{comment}

\newcommand{\on}{\operatorname}
\newcommand{\I}{\mathfrak{I}}
\newcommand{\A}{\mathfrak{A}}

\begin{document}

\section{Syntax and Semantics}
$e$ is some expression.
$v$ is some value.
$T$ is some type.
$x$ is some variable.
rule names ending with a $*$ are optional and could be convenient but will often cause nondeterminism.
rule names ending with a $'$ introduce optional features but might cause some problems.
rules for different things might have the same name,
the section titles should be interpreted as namespaces.

\subsection{Boolean}
\[e ::= true | false | ¬ e | e ∧ e\]
\[v ::= true | false\]
\[T ::= Bool\]
\subsubsection{Contraction}

\infer[(NotTrue)]{¬true → false}{} \vspace{1em}
\infer[(NotFalse)]{¬false → true}{} \vspace{1em}
\infer[(AndTrue)]{true ∧ e → e}{} \vspace{1em}
\infer[(AndTrue*)]{e ∧ true → e}{} \vspace{1em}
\infer[(AndFalse)]{false ∧ e → false}{} \vspace{1em}
\infer[(AndFalse*)]{e ∧ false → false}{} \vspace{1em}

\subsubsection{Congruence}
\infer[(Not)]{¬e → ¬e'}{e → e'} \vspace{1em}
\infer[(And)]{e ∧ e₀ → e' ∧ e₀}{e → e'} \vspace{1em}
\infer[(And*)]{e₀ ∧ e → e₀ ∧ e'}{e → e'} \vspace{1em}

\subsubsection{Types}
\infer[(True)]{true: Bool}{}  \vspace{1em}
\infer[(False)]{false: Bool}{} \vspace{1em}
\infer[(Not)]{¬e: Bool}{e: Bool} \vspace{1em}
\infer[(And)]{e₁ ∧ e₂: Bool}{e₁: Bool \hspace{1em} e₂: Bool} \vspace{1em}


\subsection{Arithmetic}
\[e ::= <n> | e + e\]
\[v ::= <n> \]
\[T ::= Num\]
\subsubsection{Contraction}

\infer[(Plus)]{<n> + <m> → <n + m>}{} \vspace{1em}

\subsubsection{Congruence}
\infer[(PlusLeft)]{e + e₀ → e' ∧ e₀}{e → e'} \vspace{1em}
\infer[(PlusRight)]{v + e → v + e'}{e → e'} \vspace{1em}
\infer[(PlusRight*)]{e₀ + e → e₀ + e'}{e → e'} \vspace{1em}

\subsubsection{Types}
\infer[(Num)]{<n>: Num}{}  \vspace{1em}
\infer[(Plus)]{e₁ + e₂: Num}{e₁: Num \hspace{1em} e₂: Num} \vspace{1em}

\subsection{Conditionals}
These kinda require Arithmetics and Booleans as a base.
\[e ::= e == e | e ? e : e\]
\subsubsection{Contraction}

\infer[(EqualTrue)]{<n> == <n> → true}{} \vspace{1em}
\infer[(EqualTrue')]{v == v → true}{} \vspace{1em}
\infer[(EqualFalse)]{<n> == <m> → false}{n \ne m} \vspace{1em}
\infer[(EqualFalse')]{ v₁ == v₂ → false}{v₁ \ne v₂} \vspace{1em}
\infer[(IfTrue)]{true ? e₁ : e₂ → e₁}{} \vspace{1em}
\infer[(IfFalse)]{false ? e₁ : e₂ → e₁}{} \vspace{1em}

\subsubsection{Congruence}
\infer[(EqualLeft)]{e == e₀ → e' ∧ e₀}{e → e'} \vspace{1em}
\infer[(EqualRight)]{v == e → v == e'}{e → e'} \vspace{1em}
\infer[(EqualRight*)]{e₀ == e → e₀ == e'}{e → e'} \vspace{1em}
\infer[(If)]{e ? e₁ : e₂ → e' ? e₁ : e₂}{e → e'} \vspace{1em}

\subsubsection{Types}
\infer[(Equal)]{e₁ == e₂: Bool}{e₁: Num \hspace{1em} e₂: Num} \vspace{1em}
\infer[(Equal')]{e₁ == e₂: Bool}{e₁: T \hspace{1em} e₂: T} \vspace{1em}
\infer[(If)]{e ? e₁ : e₂: T}{e: Bool \hspace{1em} e₁: T \hspace{1em} e₂: T} \vspace{1em}
\infer[(IfTrue')]{e ? e₁ : e₂: T}{e →* true \hspace{1em} e₁: T} \vspace{1em}
\infer[(IfFalse')]{e ? e₁ : e₂: T}{e →* false \hspace{1em} e₂: T} \vspace{1em}

\subsection{Functions}
\[e ::= λx. e | e e | x\]

\subsubsection{Contraction}
\infer[(Application)]{(λx. e) v → e[x/v]}{} \vspace{1em}
\infer[(Application*)]{(λx. e₁) e₂ → e₁[x/e₂]}{} \vspace{1em}

\subsubsection{Congruence}
\infer[(ApplicationFunction)]{e e₀ → e' e₀}{e → e'} \vspace{1em}
\infer[(ApplicationArgument)]{v e → v e'}{e → e'} \vspace{1em}
\infer[(ApplicationArgument*)]{e₀ e → e₀ e'}{e → e'} \vspace{1em}
\infer[(Abstraction')]{(λx. e) → (λx. e')}{e → e'} \vspace{1em}


\subsubsection{Types}
\infer[(Equal)]{e₁ == e₂: Bool}{e₁: Num \hspace{1em} e₂: Num} \vspace{1em}
\infer[(Equal')]{e₁ == e₂: Bool}{e₁: T \hspace{1em} e₂: T} \vspace{1em}
\infer[(If)]{e ? e₁ : e₂: T}{e: Bool \hspace{1em} e₁: T \hspace{1em} e₂: T} \vspace{1em}
\infer[(IfTrue')]{e ? e₁ : e₂: T}{e →* true \hspace{1em} e₁: T} \vspace{1em}
\infer[(IfFalse')]{e ? e₁ : e₂: T}{e →* false \hspace{1em} e₂: T} \vspace{1em}

\end{document}
